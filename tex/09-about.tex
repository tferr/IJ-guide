
\section*{Colophon\label{sec:About-this-Guide}}

The initial contents of this guide have been retrieved in 2009 from
the \href{http://imagej.nih.gov/ij/}{ImageJ website} using \href{http://www.mbayer.de/html2text/}{html2text}.
Since then, it has been complemented and updated using informations
posted on the \href{http://imagej.nih.gov/ij/list.html}{ImageJ mailing list},
\href{http://imagejdocu.tudor.lu/doku.php?id=start}{ImageJ Documentation Portal},
\href{http://fiji.sc/wiki/index.php/Fiji}{Fiji}, \href{http://imagej.nih.gov/ij/}{ImageJ},
and \href{http://developer.imagej.net/}{ImageJDev} websites and Tony
Collins \emph{\href{http://www.macbiophotonics.ca/imagej/}{ImageJ for Microscopy}}
manual. Nevertheless, because there has never been accompanying documentation
for some of the 350+ described commands, several sections have been
written from scratch based on the relevant \href{http://imagej.nih.gov/ij/developer/source/index.html}{ImageJ source code}\emph{
}and authors own experience. Legacy nomenclature that became obsolete
with version \theguideversion{} has been intentionally omitted. 

\noindent The guide was typeset with \TeX{}Live\,2012 on Mac OS\,10.6.8.
All illustrations were created with ImageJ/Fiji, loaded with G.\ Landini's
\href{http://fiji.sc/wiki/index.php/IJ_Robot}{IJ Robot} and J.\ Schindelin's
\href{http://fiji.sc/wiki/index.php/Tutorial_Maker}{Tutorial Maker}
plugins. Screenshots were produced by the \code{\noindent screencapture}
shell utility controlled by the following IJ macro: 
\begin{lstlisting}[otherkeywords={=},showstringspaces=false,tabsize=4]
 exec("screencapture -ciWo > /dev/null 2>&1 &"); run("System Clipboard");
 setLineWidth(1); setForegroundColor(111, 121, 132);
 drawRect(0, 0, getWidth, getHeight);
\end{lstlisting}


\noindent The HTML version was produced with \href{http://elyxer.nongnu.org/}{eLyXer 1.2}
and formatted using CSS code from Alex Fern�ndez and Michael H�neburg;
JavaScript code from Ciar�n O'Kelly, Stuart Langridge and Tiago Ferreira.
It uses \href{http://alexgorbatchev.com/SyntaxHighlighter/}{SyntaxHighlighter}
and icons from the \href{http://tango.freedesktop.org/Tango_Desktop_Project}{Tango Desktop Project}.


\subsection*{Getting Involved\label{sub:Getting-Involved}}

Your help is needed to improve ImageJ. Even if you are not a programmer,
your participation is important:
\begin{itemize}
\item Are you a skilled ImageJ user?\\
You might want to help out with the documentation effort: Write a
FAQ, How-To, Tutorial or \href{http://imagejdocu.tudor.lu/doku.php?id=video:start}{Video Tutorial}
on the \href{http://imagejdocu.tudor.lu/doku.php?id=howto:general:how_to_use_this_documenation_wiki}{ImageJ Documentation Portal};
Help us updating the ImageJ User Guide; Share the add-ons you may
have created with the community;  Subscribe the \href{http://imagej.nih.gov/ij/list.html}{mailing list}
and help answering the questions raised by other users.
\item Are you know knowledgeable in image processing?\\
Please join the \href{http://imagej.nih.gov/ij/list.html}{mailing list}
and participate in the discussions. You could also write a Tutorial
on the \href{http://imagejdocu.tudor.lu/doku.php?id=howto:general:how_to_use_this_documenation_wiki}{ImageJ Documentation Portal}.
\item Do you have a strong scientific background?\\
Frequently, the most demanding tasks in scientific image processing
relate to experimental design. Even if you do not have special expertise
in image processing, by participating on the \href{http://imagej.nih.gov/ij/list.html}{mailing list}
discussions, your experience will be valuable to others.
\item Do you want ImageJ to improve?\\
You can report bugs or request new features using the \href{http://imagej.nih.gov/ij/list.html}{mailing list}.
\item Do you have experience in graphic/web design?\\
If you are able to to improve the look and feel of the guide we welcome
your skills.
\end{itemize}

\subsection*{The ImageJ Icon\label{sec:About-the-Cover}}

The \index{Hartnack@Hartnack \see{IJ\, icon,}}\index{IJ icon}Hartnack
microscope (\emph{ca}.\ 1870's) depicted on the front page inspired
the \href{http://imagej.nih.gov/ij/docs/install/osx.html\#icon}{ImageJ icon for Mac OS X}.
It is based on a \href{http://www.arsmachina.com/s-hart1209.htm}{photograph}
by \href{http://www.tomgrill.com/}{Tom Grill} at \href{http://www.arsmachina.com/}{arsmachina.com}. 

Edmund Hartnack (1826--1891) was a renown microscope manufacturer
that pioneered the use of correction collars in water-immersion lenses
and the adoption of the substage condenser%
\footnote{Merico, G. Microscopy in Camillo Golgi's times.\emph{ J Hist Neurosci,}
(2003) 8(2):113--20%
}. The precision and robustness of Hartnack optics played a pivotal
role in the groundbreaking research by the Nobel laureates Robert
Koch%
\footnote{Brock, TD\emph{. }Robert Koch, A life in medicine and\emph{ }bacteriology.\emph{
ASM,} 1999%
}, Camillo Golgi%
\footnote{Brenni P. Gli strumenti di fisica dell'Istituto Tecnico Toscano --
Ottica. \emph{IMSS}, 1995 %
} and Santiago Ram�n y Cajal%
\footnote{DeFelipe \& Jones. Santiago Ram�n y Cajal and methods in neurohistology.
\emph{Trends Neurosci}, (1992) 15(7):237--46%
}.

\clearpage{}

\pdfbookmark[1]{Revision History}{revisionhistory}


\section*{Document Revision History\label{sec:Document-Revision-History}}

\noindent \begin{center}
\begin{tabular}{>{\raggedright}p{0.19\columnwidth}>{\raggedright}p{0.72\columnwidth}}
\toprule 
\addlinespace
\multicolumn{1}{c}{Date} & \multicolumn{1}{c}{Notes}\tabularnewline
\midrule
\addlinespace
\monthname  \ 2012 & Updated for v\,1.46\\
Second major revision\\
Improvements in layout and browsability\tabularnewline\addlinespace
\addlinespace
September 2011 & Updated for v\,1.45\\
Deeply revised edition with several new sections in Parts I--IV\\
Available as printable booklets\\
Redesigned HTML version \tabularnewline\addlinespace
January 2011 & Updated for v\,1.44\\
New sections on advanced ImageJ usage, Fiji scripting, command line
usage and interoperability with other software packages\tabularnewline\addlinespace
May 2010 & First HTML version\tabularnewline\addlinespace
April 2010 & First edition covering v\,1.43\tabularnewline\addlinespace
\bottomrule
\end{tabular}
\par\end{center}

\bigskip{}



\subsection*{Acknowledgments}

We are extremely grateful to Alex Fern�ndez for his wonderful \href{http://elyxer.nongnu.org/}{eLyXer},
Alex Gorbatchev for \href{http://alexgorbatchev.com/SyntaxHighlighter/}{SyntaxHighlighter}
and Johannes Schindelin for fruitful discussions and assistance with
the \href{http://fiji.sc/guide.git}{Git repository}. We are also
thankful to all of those who submitted material, criticisms, suggestions
and encouragement to update this edition. But above all, our thanks
go to the extraordinary ImageJ community for its dedication to the
project.
